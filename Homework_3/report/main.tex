\documentclass[11pt, titlepage]{article} 

\usepackage{geometry}
\geometry{left=0.8in, right=0.8in, top=0.8in, bottom=0.8in}

\usepackage{color}
\usepackage{graphicx} % slike
\usepackage{caption}
\usepackage{subcaption}
\usepackage{siunitx} % enote
\usepackage{amsmath} % enačbe
\usepackage{amssymb} % simboli
\usepackage[thinc]{esdiff} % za odvode
\usepackage[colorlinks=true]{hyperref} % za linke
\usepackage{float} % za fiksiranje figur
\usepackage{mathtools}
\usepackage{esint} % za fancy integrale
\usepackage[numbib,nottoc]{tocbibind}
\usepackage[slovene]{babel}
\usepackage{diagbox}

\usepackage{tikz}

\newcommand{\dif}[1]{\,{\rm d}#1}

\begin{document}

\begin{titlepage}
    \begin{center}
        \includegraphics[width=0.5\textwidth]{figures/FRI_logo.png}\\
        \vspace{0.5cm}
        \vspace{3cm}
        {\LARGE \bf Obhod Lune} \\
        \vspace{0.3cm}
        \vspace{2.0cm}
        {\large Numerična matematika}\\
        \vspace{0.2cm}
        {|}\\
        \vspace{0.2cm}
        {\large 3. domača naloga}\\
        \vspace{2.0cm}
    \end{center}
    \vfill
    \begin{flushleft}
        {\normalsize {\sf Avtor:} Vito Levstik\\}
    \end{flushleft}
    \vspace{2cm}
    \begin{center}
        {\normalsize \sc Akademsko leto 2024/2025}
    \end{center}
\end{titlepage}

\newpage

\section{Uvod}

Cilj naloge je simulacija potovanja vesoljskega plovila okoli Lune na tiru z vrnitvijo brez potiska.
Rešujemo omejen krožni problem treh teles, kjer je masa plovila zanemarljiva v primerjavi z maso Zemlje in Lune.
Enačbe gibanja rešujemo v vrtečem se koordinatnem sistemu, v katerem Zemlja in Luna mirujeta. Rešiti moramo torej sistem 
\begin{align*}
\ddot{x} &= x + 2\dot{y} - \frac{1 - \mu}{R^3}(x + \mu) - \frac{\mu}{r^3}(x - (1 - \mu)) \\
\ddot{y} &= y - 2\dot{x} - \frac{1 - \mu}{R^3}y - \frac{\mu}{r^3}y \\
\ddot{z} &= - \frac{1 - \mu}{R^3}z - \frac{\mu}{r^3}z
\end{align*}
kjer je $R = \sqrt{(x + \mu)^2 + y^2 + z^2}$ in $r = \sqrt{(x - (1 - \mu))^2 + y^2 + z^2}$ razdalja do Zemlje in Lune.
Parameter $\mu$ je definiran kot $\mu = \frac{m}{M+m}$, kjer je $m$ masa Lune, $M$ pa masa Zemlje, kar nam da približno vrednost $\mu = 0.01215$. V brezdimenzijskih koordinatah je Zemlja v točki
$(-\mu, 0, 0)$, Luna pa v točki $(1 - \mu, 0, 0)$.

\section{Implementacija}
Definiramo vektor stanja $X = (x, y, z, \dot{x}, \dot{y}, \dot{z})$. Njegov odvod je tako $\dot{X} = (\dot{x}, \dot{y}, \dot{z}, \ddot{x}, \ddot{y}, \ddot{z})$, kjer poznamo eksplicitne izraze za $\ddot{x}$, $\ddot{y}$ in $\ddot{z}$.
Nalogo rešimo iterativno s pomočjo Runge-Kutta metode 4. reda, ki vsak naslednji korak izračuna kot
\begin{align*}
k_1 &= h \cdot f(t_n, X_n) \\
k_2 &= h \cdot f(t_n + \frac{h}{2}, X_n + \frac{k_1}{2}) \\
k_3 &= h \cdot f(t_n + \frac{h}{2}, X_n + \frac{k_2}{2}) \\
k_4 &= h \cdot f(t_n + h, X_n + k_3) \\
X_{n+1} &= X_n + \frac{1}{6}(k_1 + 2k_2 + 2k_3 + k_4)
\end{align*}
kjer je $f(t, X)$ funkcija, ki vrača odvod vektorja stanja $X$ ob času $t$ in $h$ velikost koraka. Metodo smo implementirali v funkciji \texttt{rk4\_method}, ki sprejme začetni vektor stanja, časovni korak in število iteracij. Na vsakem koraku funkcija dodatno preveri,
če je plovilo trčilo v Luno ali Zemljo in predčasno konča simulacijo, če je prišlo do trka. Za Zemljo in Luno smo privzeli, da sta popolni krogli z radijema $R_Z = 6400 \text{ km}$ in $R_L = 1700 \text{ km}$.

Najtežji del naloge je izbor začetnih vrednosti, saj želimo najti orbite z vrnitvijo brez potiska, kar se zgodi le v redkih primerih.
Na sliki \ref{fig:theory_orbits} so prikazane orbite, dobljene iz literature \cite{XIYUN2013183}, ki jih želimo v tej nalogi reproducirati.

\section{Rezultati}
Tekom simulacij se omejimo na gibanje v $xy$ ravnini. Takšna obravnava je smiselna za majhne odmike $z$.
Za majhne odmike $z$ sta $R$ in $r$ funkciji le $x$ in $y$, torej lahko pišemo
$$
\ddot{z} = - a^2 \cdot z,
$$
kjer definiramo $a^2 = \frac{1 - \mu}{R^3} + \frac{\mu}{r^3}$. Dobimo enačbo harmoničnega nihanja, katere rešitev je
$$
z(t) = A \cos(at) + B \sin(at),
$$
kjer sta $A$ in $B$ konstanti, določeni iz začetnih pogojev. Gibanje v $z$ komponenti bo tako podobno harmoničnemu nihanju, ki se mu sproti spreminja frekvenca nihanja. 

\begin{figure}[H]
    \centering
    \begin{subfigure}[b]{0.45\textwidth}
        \centering
        \includegraphics[width=\textwidth]{figures/theory_orbit_1.png}
        \caption{Progradna orbita z začetkom na nasprotni strani Zemlje kot Luna.}
        \label{fig:orbit_1}
    \end{subfigure}
    \hfill
    \begin{subfigure}[b]{0.45\textwidth}
        \centering
        \includegraphics[width=\textwidth]{figures/theory_orbit_2.png}
        \caption{Retrogradna orbita z začetkom na nasprotni strani Zemlje kot Luna.}
        \label{fig:orbit_2}
    \end{subfigure}

    \vspace{0.5cm}

    \begin{subfigure}[b]{0.45\textwidth}
        \centering
        \includegraphics[width=\textwidth]{figures/theory_orbit_3.png}
        \caption{Progradna orbita z začetkom na isti strani Zemlje kot Luna.}
        \label{fig:orbit_3}
    \end{subfigure}
    \hfill
    \begin{subfigure}[b]{0.45\textwidth}
        \centering
        \includegraphics[width=\textwidth]{figures/theory_orbit_4.png}
        \caption{Retrogradna orbita z začetkom na isti strani Zemlje kot Luna.}
        \label{fig:orbit_4}
    \end{subfigure}

    \caption{Orbite, dobljene iz literature \cite{XIYUN2013183}.}
    \label{fig:theory_orbits}
\end{figure}


Veliko vesoljskih plovil je izstreljenih iz nižje Zemljine orbite (Low Earth Orbit), zato smo privzeli, da je plovilo izsteljeno iz višine $200 \text{ km}$ nad površjem. Vektor hitrosti plovila je v tem trenutku pravokoten na vektor pozicije, relativno na središče Zemlje. Iskanje ustreznih začetnih pogojev
je tako omejeno na iskanje začetnega kota $\alpha$ (merjen od zveznice med Zemljo in Luno v pozitivni smeri) in velikosti hitrosti plovila $v_0$. S poiskušanjem različnih kombinacij smo našli željene štiri orbite. Pri tem smo želeli doseči, da se plovilo med letom čimbolj približa Luni, za katero smo vzeli, da ima polmer $1700 \text{ km}$. Parametri začetnih pogojev
so prikazani v tabeli \ref{tab:initial_conditions}.
Enote za hitrost v tabeli \ref{tab:initial_conditions} so brezdimenzijske. Razdalja med Zemljo in Luno je v brezdimenzijskih enotah 1, kar ustreza približno $384400 \text{ km}$. Podobno je kotna hitrost vrtenja sistema enaka 1. 
Ko uporabimo dejstvo, da je perioda vrtenja Lune okoli Zemlje približno $2.36 \cdot 10^6 \text{ s}$, dobimo, da je brezdimenzijska enota hitrosti enaka $1.023 \text{ km/s}$.
Da dobimo hitrosti v realnih enotah, moramo brezdimenzijske hitrosti pomnožiti s to vrednostjo.
\begin{table}[H]
    \centering
    \begin{tabular}{|c|c|c|c|c|}
        \hline
        \textbf{Št.} & $\alpha$ [°] & $v_0$ & $d_L$ [km] & $t$ [dni] \\
        \hline
        1 & 227.3 & 10.6459 & 1768.5 & 5.687\\
        2 & 210 & -10.6823 & 1867.7 & 5.665\\
        3 & -5 & 10.64555 & 1842.7 & 27.46\\
        4 & 30 & -10.68257 & 1774.8 & 30.84\\
        \hline
    \end{tabular}
    \caption{Začetni pogoji za iskane štiri orbite, minimalna dobljena razdalja do Lune med simulacijo in čas potovanja. Enote za hitrost so brezdimenzijske.}
    \label{tab:initial_conditions}
\end{table}

Na sliki \ref{fig:orbits} so prikazane orbite, dobljene z uporabo začetnih pogojev iz tabele \ref{tab:initial_conditions} in z uporabo časovnega koraka $h = 10^{-6}$.
\begin{figure}[h]
    \centering
    \begin{subfigure}[b]{0.45\textwidth}
        \centering
        \includegraphics[width=\textwidth]{figures/free_return_orbit_1.png}
        \caption{Progradna orbita z začetkom na nasprotni strani Zemlje kot Luna.}
        \label{fig:orbit_1_sim}
    \end{subfigure}
    \hfill
    \begin{subfigure}[b]{0.45\textwidth}
        \centering
        \includegraphics[width=\textwidth]{figures/free_return_orbit_2.png}
        \caption{Retrogradna orbita z začetkom na nasprotni strani Zemlje kot Luna.}
        \label{fig:orbit_2_sim}
    \end{subfigure}

    \vspace{0.5cm}

    \begin{subfigure}[b]{0.45\textwidth}
        \centering
        \includegraphics[width=\textwidth]{figures/free_return_orbit_3.png}
        \caption{Progradna orbita z začetkom na isti strani Zemlje kot Luna.}
        \label{fig:orbit_3_sim}
    \end{subfigure}
    \hfill
    \begin{subfigure}[b]{0.45\textwidth}
        \centering
        \includegraphics[width=\textwidth]{figures/free_return_orbit_4.png}
        \caption{Retrogradna orbita z začetkom na isti strani Zemlje kot Luna.}
        \label{fig:orbit_4_sim}
    \end{subfigure}

    \caption{Simulirane orbite z uporabo začetnih pogojev iz tabele \ref{tab:initial_conditions}.}
    \label{fig:orbits}
    
\end{figure}
Velikost koraka $h$ določa natančnost simulacije. S preizkušanjem različnih vrednosti ugotovimo, da za prevelike vrednosti (nad $10^{-5}$) pride do kritičnih napak. Plovilo bodisi trči v Luno, ali pa se ne vrne na Zemljo.
Iz tega razloga preverimo kako se končne koordinate plovila spreminjajo za vrednosti koraka $h$ manjše od $10^{-5}$. Rezultati so prikazani na sliki \ref{fig:step_size}.
\begin{figure}[h]
    \centering
        \begin{subfigure}[b]{0.45\textwidth}
        \centering
        \includegraphics[width=\textwidth]{figures/x_coordinate_vs_dt.png}
        \caption{Končna koordinata $x$ plovila glede na velikost koraka $h$.}
        \label{fig:x_coordinate_vs_dt}
    \end{subfigure}
    \hfill
    \begin{subfigure}[b]{0.45\textwidth}
        \centering
        \includegraphics[width=\textwidth]{figures/y_coordinate_vs_dt.png}
        \caption{Končna koordinata $y$ plovila glede na velikost koraka $h$.}
        \label{fig:y_coordinate_vs_dt}
    \end{subfigure}

    \caption{Končne koordinate plovila glede na velikost koraka $h$ za primer prve orbite.}
    \label{fig:step_size}
\end{figure}
Vidimo, da koordinate konvergirajo določenim vrednostim z manjšanjem koraka, kar je pričakovano. V naših simulacijah uporabimo vrednost $h = 10^{-6}$, kar nam da rezultat koordinat na 4 decimalke natančno, kar ustreza natančnosti na $38.44 \text{ km}$.
Za potrebe naloge je to dovolj, saj so dobljene orbite primerljive z literaturo.

Za konec še pogledamo stabilnost orbit glede na začetne pogoje. Omejimo se samo na prvo orbito.
Tabela \ref{tab:min_distance} prikazuje minimalne razdalje med plovilom in Luno glede na začetne pogoje. X označuje, da je plovilo trčilo v Luno, O pa da plovilo ni padlo nazaj na Zemljo.
Iz rezultatov lahko razberemo, da že majhne spremembe v začetnih pogojih bistveno vplivajo na orbito. Čimbližji prelet Lune je težko doseči, saj z manjšanjem hitrosti dosežemo nižji prelet, vendar se lahko ob prenizki hitrosti ne vrnemo na Zemljo, ali pa celo trčimo v Luno.
\begin{table}[h]
    \centering
    \caption{Matrika minimalnih razdalj (v km) med plovilom in Luno glede na začetni kot $\alpha$ in začetno hitrost $v_0$.}
    \label{tab:min_distance}
    \begin{tabular}{|c|rrrrrrrrr|}
        \hline
        \diagbox{$\alpha$ [°]}{$v_0$} & 10.6455 & 10.6456 & 10.6457 & 10.6458 & 10.6459 & 10.6460 & 10.6461 & 10.6462 & 10.6463 \\
        \hline
        227.26 & X & X & X & X & X & X & O & 1802.22 & 1857.76 \\
        227.27 & X & X & X & X & X & O & 1780.13 & 1835.64 & 1891.50 \\
        227.28 & X & X & X & X & O & O & 1813.38 & 1869.22 & 1925.40 \\
        227.29 & X & X & X & X & O & 1791.00 & 1846.81 & 1902.96 & 1959.45 \\
        227.30 & X & X & X & O & 1768.50 & 1824.26 & 1880.39 & 1936.86 & 1993.66 \\
        227.31 & X & X & X & O & 1801.59 & 1857.68 & 1914.13 & 1970.91 & 2028.02 \\
        227.32 & X & X & O & 1778.79 & 1834.85 & 1891.27 & 1948.03 & 2005.12 & 2062.52 \\
        227.33 & X & O & O & 1811.89 & 1868.27 & 1925.01 & 1982.08 & 2039.48 & 2097.17 \\
        227.34 & X & O & 1788.80 & 1845.15 & 1901.85 & 1958.90 & 2016.28 & 2073.98 & 2131.96 \\
        \hline
    \end{tabular}
\end{table}

\bibliographystyle{unsrt}   % You can use other styles like alpha, ieeetr, apalike, etc.
\bibliography{references}   % Don't include the .bib extension

\end{document}